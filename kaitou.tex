\documentclass[uplatex,dvipdfmx,a4paper,11pt]{jsarticle}
\usepackage{cite}

%括弧
\usepackage{delimseasy}

%二段組
\usepackage{multicol}
\setlength{\columnseprule}{.5pt} %中央の線

%フォント変更
\usepackage[T1]{fontenc}
\usepackage{tgheros}
\renewcommand{\kanjifamilydefault}{\gtdefault}
\renewcommand{\familydefault}{\sfdefault}
\usepackage[scaled]{helvet}
%数式フォント
%数式
\usepackage{nccmath,amsmath,amssymb}
\usepackage{mathtools}
\usepackage{empheq} %数式の囲いに使う
\usepackage{physics}
\usepackage{bm}
\usepackage[bbsets]{jkmath} %\Nなどをつかえる
\usepackage{amsthm}

%ハイパーリンク用
\usepackage{url}
\usepackage[dvipdfmx]{hyperref}
\usepackage{xcolor}
\hypersetup{colorlinks=true,citecolor={green!40!black},linkcolor={green!40!black},urlcolor={blue!70!black},}
\usepackage[otfcid,otfmacros]{pxjahyper}
\usepackage{bookmark}

%tcolorbox系
\usepackage[many]{tcolorbox}
\tcbuselibrary{breakable,skins}
\newtcolorbox{hosoibox}[1]{colframe=black,colback=white,coltitle=black,colbacktitle=white,boxrule=0.5pt,arc=0mm,,enhanced,attach boxed title to top left={xshift=10mm,yshift=-3mm},boxed title style={frame hidden},title=#1}

%leftbar環境に注釈が入れられないことを解消する環境.名前は,tcolorboxの[t]とleftbarの組み合わせ
\newtcolorbox{tbleftline}{blanker,left=5mm,borderline west={1.1mm}{0pt}{black}}
\newenvironment{tleftbar}{\begin{tbleftline}\setlength{\parindent}{1zw}}{\end{tbleftline}}


\usepackage {framed,color}

%題名付き四角
\usepackage{ascmac}
\usepackage{fancybox}

%囲い文字
\usepackage[multi,jis2004,uplatex,expert,deluxe]{otf}
%図に使うもの
\usepackage{tikz}
\usetikzlibrary{cd}
\usepackage[dvipdfmx,marginparwidth=0pt,margin=25truemm]{geometry}
\usepackage{bxpapersize}
\usepackage[absolute,overlay]{textpos} %図の配置を好きにする

%画像
\usepackage{wrapfig}
%footnoteの変更
\renewcommand\thefootnote{{\dag}\arabic{footnote}}
\renewcommand{\thempfootnote}{{\dag}\arabic{mpfootnote}}
\interfootnotelinepenalty=10000

\usepackage{oubraces} %overunderbraces

%underbraceの文字数が多いときのためのadunderbrace
\usepackage{ifthen}
\newlength{\wdTempA}
\newlength{\wdTempB}
\newcommand{\adunderbrace}[2]{%
\settowidth{\wdTempA}{$#1$}%
\settowidth{\wdTempB}{${\scriptstyle #2}$}%
\ifthenelse{\wdTempA<\wdTempB}{%
\hspace*{.5\wdTempA}\hspace*{-.5\wdTempB}%
\underbrace{#1}_{#2}%
\hspace*{.5\wdTempA}\hspace*{-.5\wdTempB}%
}{%
\underbrace{#1}_{#2}%
}%
}%
%丸付き文字
\newcommand{\ctext}[1]{\raise0.2ex\hbox{\textcircled{\scriptsize{#1}}}}

%ユーザー定義
\newcommand{\dash}[1]{#1^\prime}
\newcommand{\ddash}[1]{#1^{\prime\prime}}
\newcommand{\dddash}[1]{#1^{\prime\prime\prime}}
\newcommand{\hodash}[2]{#2^{(#1)}}
\renewcommand{\labelenumi}{(\arabic{enumi})}%itemを(数字)に変更

%inputなどのプリアンブルを無視
\usepackage{docmute}

\begin{document}

\title{雪江代数学1(群論入門):解答集}
\author{}
\date{\today}
\maketitle
\begin{multicols*}{3}
    \tableofcontents
\end{multicols*}
\addcontentsline{toc}{section}{\texorpdfstring{目次}{目次}}
\newpage

\section*{第1章の演習問題}
\addcontentsline{toc}{section}{\texorpdfstring{第1章の演習問題}{第1章の演習問題}}

\subsection*{1.1.1}
\addcontentsline{toc}{subsection}{\texorpdfstring{1.1.1}{1.1.1}}
\begin{tleftbar}
    $f$が$g$,$A$が$X$,$B$が${}^t X X$にそれぞれ対応する.
\end{tleftbar}

\subsection*{1.1.2-(1)}

\addcontentsline{toc}{subsection}{\texorpdfstring{1.1.2-(1)}{1.1.2-(1)}}
\begin{tleftbar}
    $f(S)= \{ 3, 4 \}$
\end{tleftbar}

\subsection*{1.1.2-(2)}
\addcontentsline{toc}{subsection}{\texorpdfstring{1.1.2-(2)}{1.1.2-(2)}}
\begin{tleftbar}
$f^{-1}(S_1)= \varnothing$,$f^{-1}(S_2)=\{ 1 , 3 , 4 , 5 \}$
\end{tleftbar}

\subsection*{1.1.2-(3)}
\addcontentsline{toc}{subsection}{\texorpdfstring{1.1.2-(3)}{1.1.2-(3)}}
\begin{tleftbar}
$2 \in B$であるが,$f(2)= \varnothing$であるため,$f$は全射でない.
\end{tleftbar}

\subsection*{1.1.2-(4)}
\addcontentsline{toc}{subsection}{\texorpdfstring{1.1.2-(4)}{1.1.2-(4)}}
\begin{tleftbar}
$f(3)= f(5)$であるが,$3 \ne 5$であるため,$f$は単射でない.
\end{tleftbar}

\newpage 

\section*{第2章}
\addcontentsline{toc}{section}{\texorpdfstring{第2章}{第2章}}

\subsection*{2.3.3}
\addcontentsline{toc}{subsection}{\texorpdfstring{2.3.3}{2.3.3}}
\begin{tleftbar}
    \begin{proof}
        $H_1,H_2$は$G$の部分群ゆえ命題2.3.2より,$1_{G}\in H_1$かつ$1_{G}\in H_2$.したがって,$1_G\in H_2\cap H_2$.さらに,$a,b \in H_1\cap H_2$であるとき,$a,b\in H_1$であるから,命題2.3.2より$ab\in H_1$.同様にして$ab\in H_2$.したがって,$ab\in H_1\cap H_2$.$a\in H_1\cap H_2$とすると,$a\in H_1$と命題2.3.2より$a^{-1}\in H_1$.同様に$a\in H_2$より$a^{-1}\in H_2$.したがって,$a^{-1}\in H_1 \cap H_2$.以上より,命題2.3.2の(1)(2)(3)を$H_1\cap H_2$は満たすから,$H_1\cap H_2$は$G$の部分群.
    \end{proof}
\end{tleftbar}


\subsection*{2.3.14}
\addcontentsline{toc}{subsection}{\texorpdfstring{2.3.14}{2.3.14}}
\begin{tleftbar}
    \begin{proof}
        $S_1 \subset S_2$ならば,$S_1$のすべての元を$S_2$が含むので,$S_1$の元による語はすべて,$S_2$の元から作れる.したがって,$\langle S_1 \rangle \subset \langle S_2 \rangle$
    \end{proof}
\end{tleftbar}

\subsection*{2.3.15}
\addcontentsline{toc}{subsection}{\texorpdfstring{2.3.15}{2.3.15}}
\begin{tleftbar}
    $\langle S \rangle =n\Z$というのは,$\Z$は演算を加法とする群であるから,$x^n=\underbrace{x+\cdots+x}_{n個}=nx$ということになり,$\langle S \rangle =\{x^n\mid n\in \Z\}=\{nx\mid n\in \Z\}$となる.\footnote{冪$x^n$の定義は定義2.1.3による.}
\end{tleftbar}

\subsection*{2.3.20-(3)}
\addcontentsline{toc}{subsection}{\texorpdfstring{2.3.20-(3)}{2.3.20-(3)}}

\begin{tleftbar}
    $S=\{\sigma,\tau\}$とすると,$\{\sigma\}\subset S$かつ$\{\tau\}\subset S$と命題2.3.14より,$\langle {\sigma} \rangle \subset \langle S \rangle$かつ$\langle {\tau} \rangle \subset \langle S \rangle$.したがって$\langle {\sigma} \rangle  \cup \langle {\tau} \rangle \subset \langle S \rangle$となり,(1),(2)の結果と合わせて,回答のようになる. 
\end{tleftbar}


\subsection*{2.3.22}
\addcontentsline{toc}{subsection}{\texorpdfstring{2.3.22}{2.3.22}}

\begin{tleftbar}
    $j=1,\cdots,t$に対し写像
    \begin{align*}
        i_j:G_j\ni g_j \mapsto (1_{G_1},\cdots,1_{G_{j-1}},g_j,1_{G_{j+1}},\cdots,1_{G_t}) \in G_1\cross \cdots \cross  G_t
    \end{align*}  
    を考えると,これは単射であるから,$G_j$を$G_1\cross \cdots \cross  G_t$の部分集合とみなせるというのは,写像の終域$G_1\cross \cdots \cross  G_t$の部分集合の元に対して,$G_j$の元がただ一つ対応するということから,$G_j$を$G_1\cross \cdots \cross  G_t$の部分集合と``みなす''ことができるということ.
\end{tleftbar}


\subsection*{2.4.4}
\addcontentsline{toc}{subsection}{\texorpdfstring{2.4.4}{2.4.4}}

%なるべく正しそうな証明にしようと思って,グダグダ書いた.

\begin{tleftbar}
    \begin{proof}
        $ n\in \Z$に対して$r(n)$を以下で定義する.\footnote{$j=m$だったりすると,$i_{j+1}$に移るわけではないので,そういったものを防ぐために$r$を導入した.}
        \begin{align*}
            r(n)=
            \begin{cases}
                (\text{$n$を$m$で割った余り})&(n\not\in m\Z)\\
                m &(n\in m\Z)
            \end{cases}
        \end{align*}
        $\sigma=(i_1\cdots i_m)$とすると,$\sigma$により$i_j$は$i_{r(j+1)}$に移る.ある$n\in \N$に対して$\sigma^n$によって$i_j$は$i_{r(j+n)}$に移るとすると,$\sigma^{n+1}$によって$i_j$は$i_{r(j+n+1)}$に移る.数学的帰納法により,$\sigma^n$によって$i_j$は$i_{r(j+n)}$に移る.ここで,$j=r(j+n)$を満たす最小の自然数$n$は$m$である.これは,$1\leq j \leq m$の任意の$j$に対して成立する.したがって,$\sigma$の位数は$m$である.
    \end{proof}
\end{tleftbar}


\subsection*{2.4.17}
\addcontentsline{toc}{subsection}{\texorpdfstring{2.4.17}{2.4.17}}
\begin{tleftbar}
    ここでの群は,加法による群である.$d\in H$ならば$d+d\in H$である.これを繰り返せば$d^q=qd\in H$である.$H$は$\Z$の部分群であるから,$(d^q)^{-1}=-qd\in H$である.よって,$n\in H$ならば$r=n+(-qd)=n-qd\in H$である.ところで,$r$は$0\leq r<d$を満たすものであるので,$r\neq 0$とすると$d$未満の正整数が$H$の元としてあることになって,$d$の取り方に矛盾する.よって,$r=0$であるから,$n=qd\in d\Z$となり,$n\in H$ならば$n\in d\Z$である.したがって,$H \subset d\Z$.\footnote{$H\supset d\Z$は明らかなのか,証明が省かれている.}

    \begin{proof}
        ($H \supset d\Z$の証明)\,\,$n\in d\Z$とすると,$d\Z=\langle \{d\}\rangle$であり,$d\in H$ゆえ,$\{d\}\subset H$である.また,命題2.3.13より$\{d\}\subset H$ならば$\langle \{d\}\rangle \subset H$である.したがって,$d\Z \subset H$である.
    \end{proof}
    $H \subset d\Z$かつ$H \supset d\Z$より$H=d\Z$
\end{tleftbar}

\section*{第2章の演習問題}
\addcontentsline{toc}{section}{\texorpdfstring{第2章の演習問題}{第2章の演習問題}}

\subsection*{2.1.1}
\addcontentsline{toc}{subsection}{\texorpdfstring{2.1.1}{2.1.1}}
\begin{tleftbar}
    $1$が単位元である.$0$に逆元がないことがわかる.したがって,$G$は演算$\cdot$により群とならない.
\end{tleftbar}


\subsection*{2.1.2}
\addcontentsline{toc}{subsection}{\texorpdfstring{2.1.2}{2.1.2}}
\begin{tleftbar}
    $0$が単位元である.$a+b+ba=0$とすると,$a\neq -1$のときは$b=-\frac{a}{1+a}$となるが,$a=-1$のときは任意の$b\in \R$に対して$a+b+ab=-1$となるため,$-1$の逆元が存在しない.したがって,$\R$は演算$\circ$により群とならない.\footnote{結合法則は成立している.}
\end{tleftbar}

\subsection*{2.1.4}
\addcontentsline{toc}{subsection}{\texorpdfstring{2.1.4}{2.1.4}}
\begin{tleftbar}
    $((ab)c)d=(a(bc))d=a((bc)d)$
\end{tleftbar}


\subsection*{2.2.2}
\addcontentsline{toc}{subsection}{\texorpdfstring{2.2.2}{2.2.2}}
\begin{tleftbar}
    \begin{description}
        \item[(3)]$39$を法とする合同式を使うと 
        \begin{align*}
            16^8&=(13+3)^8\\
            &\equiv 13^8+3^8 &&(\text{これら以外の項は$13$と$3$の両方を因数に持つ})\\
            &\equiv 13(12+1)^7+3^2\cdot 27\cdot 27\\
            &\equiv 13 +3^2(13\cdot 2 +1)(13\cdot 2 +1) &&((12+1)^7\text{を展開すると,$1$以外の項は全て$3$を因数に持つ})\\
            &\equiv 13+3^2\cdot 1 =22
        \end{align*} 
        となる.ここから答えがわかる.
        \item[(4)] (3)と同様に計算を行うと
        \begin{fleqn}[30pt]
            \begin{align*}
                16^{34}&=(13+3)^{34}\\
                &\equiv 13^{34}+3^{34} \\
                &\equiv 13(12+1)^{33}+3(13\cdot 2+1)^{11}\\
                &\equiv 13+3=16
            \end{align*}  
        \end{fleqn} 
        となる.ここから答えがわかる.
    \end{description}
\end{tleftbar}


\subsection*{2.3.1}
\addcontentsline{toc}{subsection}{\texorpdfstring{2.3.1}{2.3.1}}
\begin{tleftbar}
\begin{proof}
        $H$が$G$の部分群であることと同値な条件は命題2.3.2から,
        \begin{fleqn}[30pt]
            \begin{align*}
                \begin{cases}
                    \text{\ajMaru{1}}\quad  1_G\in H\\
                    \text{\ajMaru{2}}\quad \forall x,\forall y\in H\,,\,xy\in H\\
                    \text{\ajMaru{3}}\quad \forall x\,,\, x^{-1}\in H
                \end{cases}
            \end{align*}
        \end{fleqn}
        である.これを用いて証明する.\\
        $\Longrightarrow$の証明$\colon$ \ajMaru 2 と\ajMaru 3 より $H$が$G$の部分群であれば,任意の$x,y\in H$に対して$x^{-1}y\in H$\\
        $\Longleftarrow$の証明$\colon$ 任意の$x\in H$に対して$x^{-1}x=1_G\in H$である(\ajMaru 1).$1_G \in H$より,任意の$x\in H$に対して$x^{-1}1_G=x^{-1}\in H$である(\ajMaru 3).任意の$x\in H$に対して$x^{-1}\in H$であるから,任意の$x,y\in H$に対して$(x^{-1})^{-1}y=xy\in H$である(\ajMaru 2).
\end{proof}
\end{tleftbar}



\subsection*{2.3.2}
\addcontentsline{toc}{subsection}{\texorpdfstring{2.3.2}{2.3.2}}
\begin{tleftbar}
    \begin{proof}
        まずは,命題2.3.2を使って考えてみる.$G=\mathrm{GL}_{2n}(\R)$とする.\footnote{P31例2.3.9によると,$\mathrm{Sp}(2n)=\mathrm{Sp}(4n,\R)$となるはずだが,$\mathrm{GL}_{2n}(\R)$の部分群になるためにはそんな訳ないので,ここでは$\mathrm{Sp}(2n)=\mathrm{Sp}(2n,\R)$と考える.}
        単位行列$I_{2n}=1_G\in G$は\\
        ${}^tI_{2n}J_nI_{2n}=J_n$を満たすから,$1_G \in \mathrm{Sp}(2n)$である.また,$A,B\in \mathrm{Sp}(2n)$とすると,
        \begin{align*}
            ^t(AB)J_n(AB)={}^tB{}^tAJ_nAB ={}^tBJ_nB=J_n
        \end{align*}
        となるから,$AB\in \mathrm{Sp}(2n)$である.また,$A\in \mathrm{Sp}(2n)$とすると
        \begin{align*}
            ^t(A^{-1})J_nA^{-1}=({}^tA)^{-1}\textcolor{red}{J_n}A^{-1}=({}^tA)^{-1}\textcolor{red}{{}^tAJ_nA}A^{-1}=J_n
        \end{align*}
        となるから,$A^{-1}\in \mathrm{Sp}(2n)$である.
    \end{proof}
    \begin{proof}
        次に,演習問題2.3.1の必要十分条件を使って考えてみる.$1_G \in \mathrm{Sp}(2n)$より,$\mathrm{Sp}(2n)$は空でない$G$の部分集合である.$A,B\in \mathrm{Sp}(2n)$とすると
        \begin{align*}
            {}^t(A^{-1}B)\textcolor{red}{J_n}(A^{-1}B)={}^tB {}^t(A^{-1})\textcolor{red}{{}^t\!AJ_nA}A^{-1}B={}^tB ({}^t\!A)^{-1}\,{}^t\!A J_n B ={}^t BJ_nB=J_n
        \end{align*}
        となるから,$A^{-1}B\in \mathrm{Sp}(2n)$である.
    \end{proof}
\end{tleftbar}



\subsection*{2.3.3}
\addcontentsline{toc}{subsection}{\texorpdfstring{2.3.3}{2.3.3}}
\begin{tleftbar}
    \begin{proof}
        命題2.3.2を使って考える.$G=\mathrm{GL}_{n}(\C)$とする.単位行列$I_n=1_G\in G$は${}^t\bar{I_n}I_n={}^tI_nI_n=I_n$を満たすから,$1_G\in \mathrm{U}(n)$である.また,$A,B\in \mathrm{U}(n)$とすると,
        \begin{align*}
            {}^t(\bar{A}\bar{B})(AB)={}^t\bar{B}{}^t\!\bar{A}AB={}^t\bar{B}B=I_n
        \end{align*}
        となるから,$AB\in \mathrm{U}(n)$である.また,$A\in \mathrm{U}(n)$とすると,
        \begin{align*}
            {}^t\!\overline{A^{-1}}A^{-1}={}^t\!\overline{A^{-1}}\textcolor{red}{I_n}A^{-1}=\overline{{}^t\!A}{}^{-1}\textcolor{red}{{}^t\!\bar{A}A}A^{-1}=I_n
        \end{align*}
        となるから,$A^{-1}\in \mathrm{U}(n)$である.
    \end{proof}
    \begin{proof}
        演習問題2.3.1の必要十分条件を使って考える.$1_G\in \mathrm{U}(n)$より,$\mathrm{U}(n)$は空でない$G$の部分集合である.$A,B\in \mathrm{U}(n)$とすると
        \begin{align*}
            {}^t(\overline{A^{-1}B})(A^{-1}B)={}^t\bar{B}{}^t(\overline{A^{-1}})\textcolor{red}{I_n}A^{-1}B={}^t\bar{B}\,({}^t\bar{A}){}^{-1}\textcolor{red}{{}^t\bar{A}A}A^{-1}B={}^t\bar{B}B=I_n
        \end{align*}
        となるので,$A^{-1}B\in \mathrm{U}(n)$.ただし,共役を取ってから逆行列を求めても,逆行列を求めてから共役を取っても変わらず\footnote{行列式の計算は和と積のみで,余因子を求めるときにも和と積の計算しかしない.任意の複素数$z,w$に対して$\overline{zw}=\bar{z}\bar{w}$で,$\overline{z+w}=\bar{z}+\bar{w}$であることからこれがわかる.},共役を取ってから転置を取っても,転置をとってから共役を取っても変わらないことを用いた.
    \end{proof}
\end{tleftbar}


\end{document}